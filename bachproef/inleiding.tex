%%=============================================================================
%% Inleiding
%%=============================================================================

\chapter{\IfLanguageName{dutch}{Inleiding}{Introduction}}%
\label{ch:inleiding}


IPv6 is de opvolger van het huidige IPv4-protocol dat momenteel wordt gebruikt voor internetcommunicatie. Het groeiende aantal apparaten dat is aangesloten op het internet en de toename van de hoeveelheid data die wordt verzonden, vraagt om een efficiënter en schaalbaarder protocol. IPv6 biedt deze mogelijkheid, maar brengt ook nieuwe beveiligingsuitdagingen met zich mee. Het is daarom essentieel om onderzoek te doen naar de beveiliging van dit protocol en oplossingen te vinden om de beveiliging ervan te waarborgen.
\newline

IPv6 wordt steeds meer gebruikt en geïmplementeerd in netwerken over de hele wereld. Dit maakt het een potentieel doelwit voor aanvallers die op zoek zijn naar manieren om netwerken te compromitteren en gevoelige informatie te stelen. Het begrijpen en verbeteren van de beveiliging van IPv6 is daarom van cruciaal belang om aanvallen te voorkomen en de integriteit en vertrouwelijkheid van gegevens te beschermen.
\newline

Ten slotte is er nog niet heel veel onderzoek gedaan naar de beveiliging van IPv6 en zijn er nog steeds veel onopgeloste problemen en uitdagingen op dit gebied. Een reden hiervoor is dat IPv4 nog het meest wordt gebruikt en dat er veel aandacht nog naar IPv4 gaat. Onderzoek van de beveiliging van IPv6, waarin de nieuwste ontwikkelingen en oplossingen worden beschreven, kan bijdragen aan de ontwikkeling van effectieve beveiligingsmaatregelen en kan een belangrijke rol spelen bij de verdere ontwikkeling van dit protocol.
\clearpage


\section{\IfLanguageName{dutch}{Probleemstelling}{Problem Statement}}%
\label{sec:probleemstelling}

De overgang naar IPv6 op het internet is onvermijdelijk, maar kleinere bedrijven hebben vaak beperkte middelen om te investeren in cybersecurity. Bovendien zijn ze zich vaak niet bewust van de veiligheidsrisico's die gepaard gaan met de implementatie van IPv6. Dit maakt hen kwetsbaar voor potentiële aanvallen, met aanzienlijke schade voor zowel het bedrijf als de klanten tot gevolg. Bovendien worden deze aanvallen steeds toegankelijker door het toenemende aantal open source tools die gemakkelijk op internet te vinden zijn.

Het is belangrijk dat kleine bedrijven zich bewust zijn van de potentiële veiligheidsrisico's die gepaard gaan met IPv6. Ze moeten weten hoe ze deze risico's kunnen minimaliseren. Hoewel de implementatie van beveiligingsmaatregelen kosten met zich meebrengt, kan het niet beschermen tegen aanvallen aanzienlijk duurder zijn. Daarom moeten kleine bedrijven de noodzakelijke beveiligingsmaatregelen implementeren om hun bedrijf en klanten te beschermen.

\section{\IfLanguageName{dutch}{Onderzoeksvraag}{Research question}}%
\label{sec:onderzoeksvraag}



Grondig onderzoek naar IPv6-aanvallen is van essentieel belang om veiligheidsmaatregelen te ontwikkelen die deze aanvallen kunnen voorkomen of beperken. Het huidige onderzoek richt zich specifiek op het begrijpen van de effecten van de meest voorkomende IPv6-aanvallen, waarbij een vrij beschikbare opensource tool wordt gebruikt. Het doel is om een diepgaand inzicht te verkrijgen in de schadelijke gevolgen van deze aanvallen en hun impact op de beveiliging van netwerken en systemen. Daarnaast is het streven om optimale mitigatiemethoden te identificeren en te bevorderen, bij voorkeur met gebruikmaking van opensource tools. Het onderzoek is gericht op het vinden van de meest effectieve en efficiënte manieren om IPv6-aanvallen te voorkomen, detecteren en beperken. 

\section{\IfLanguageName{dutch}{Onderzoeksdoelstelling}{Research objective}}%
\label{sec:onderzoeksdoelstelling}

Deze thesis heeft als doelstelling om het probleem van de meest voorkomende kwetsbaarheden van IPv6 te onderzoeken en aan te pakken. Het onderzoek zal zich richten op kwetsbaarheden zoals header manipulatie, multicast-adressen, \newline TCP-aanvallen, neighbor discovery attacks en mobiliteity attacks. Bovendien zal er een proof of concept worden gepresenteerd om de impact van deze aanvallen te analyseren en de effectiviteit van de beschikbare mitigatiemaatregelen te beoordelen. Door dit te doen, zal deze studie waardevolle inzichten bieden over welke kwetsbaarheden en mitigatiemaatregelen de focus moeten hebben om de veiligheid van IPv6 te waarborgen.

\section{\IfLanguageName{dutch}{Opzet van deze bachelorproef}{Structure of this bachelor thesis}}%
\label{sec:opzet-bachelorproef}

% Het is gebruikelijk aan het einde van de inleiding een overzicht te
% geven van de opbouw van de rest van de tekst. Deze sectie bevat al een aanzet
% die je kan aanvullen/aanpassen in functie van je eigen tekst.

De rest van deze bachelorproef is als volgt opgebouwd:

In Hoofdstuk~\ref{ch:stand-van-zaken} wordt een overzicht gegeven van de stand van zaken binnen het onderzoeksdomein, op basis van een literatuurstudie.

In Hoofdstuk~\ref{ch:methodologie} wordt de methodologie toegelicht en worden de gebruikte onderzoekstechnieken besproken om een antwoord te kunnen formuleren op de onderzoeksvragen.

% TODO: Vul hier aan voor je eigen hoofstukken, één of twee zinnen per hoofdstuk
In Hoofdstuk~\ref{ch:resultaten}, worden de resultaten van de onderonderzoekstechnieken besproken.

In Hoofdstuk~\ref{ch:conclusie}, tenslotte, wordt de conclusie gegeven en een antwoord geformuleerd op de onderzoeksvragen. Daarbij wordt ook een aanzet gegeven voor toekomstig onderzoek binnen dit domein.