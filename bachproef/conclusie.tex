%%=============================================================================
%% Conclusie
%%=============================================================================

\chapter{Conclusie}%
\label{ch:conclusie}

% TODO: Trek een duidelijke conclusie, in de vorm van een antwoord op de
% onderzoeksvra(a)g(en). Wat was jouw bijdrage aan het onderzoeksdomein en
% hoe biedt dit meerwaarde aan het vakgebied/doelgroep? 
% Reflecteer kritisch over het resultaat. In Engelse teksten wordt deze sectie
% ``Discussion'' genoemd. Had je deze uitkomst verwacht? Zijn er zaken die nog
% niet duidelijk zijn?
% Heeft het onderzoek geleid tot nieuwe vragen die uitnodigen tot verder 
%onderzoek?



Het scannen van IPv6-adressen biedt een waardevol perspectief op de structuur van een netwerk en helpt bij het ontdekken van potentiële kwetsbaarheden. Door systematisch de IPv6-adressen binnen een netwerk te scannen en te analyseren, kan een beter inzicht worden verkregen in de lay-out en configuratie van het netwerk. Dit stelt netwerkbeheerders in staat om een gedetailleerd overzicht te krijgen van de verbonden apparaten, subnetten en de algehele topologie van het IPv6-netwerk.
Het scannen van IPv6-adressen biedt ook een belangrijk middel om kwetsbaarheden en beveiligingsrisico's te ontdekken. Door het uitvoeren van gerichte scans kunnen potentiële zwakke punten en blootstellingen in het netwerk worden geïdentificeerd. Dit omvat bijvoorbeeld het detecteren van open poorten, ongepatchte systemen, onveilige configuraties en andere potentiële beveiligingslekken. Het verkrijgen van dit inzicht stelt netwerkbeheerders in staat om proactief maatregelen te nemen om de beveiliging te verbeteren, kwetsbaarheden te patchen en risico's te minimaliseren.
\newline

Deze functionaliteit van de Addr6-tool biedt aanzienlijke voordelen voor het beheer en de configuratie van het IPv6-netwerk. Door gerichte filterregels in te stellen, kunnen netwerkbeheerders ervoor zorgen dat alleen het gewenste verkeer wordt toegelaten op het adres 2001:db8:acad:10::8. Dit helpt bij het handhaven van een geoptimaliseerde en veilige netwerkomgeving, waarbij ongewenst verkeer en mogelijke bedreigingen worden geweerd. Bovendien vereenvoudigt de Addr6-tool het beheerproces door een intuïtieve en gebruiksvriendelijke interface te bieden, waardoor netwerkbeheerders efficiënter kunnen werken en snel de gewenste adresconfiguratie kunnen waarborgen.
Kortom, de Addr6-tool biedt waardevolle functionaliteit voor het beheer van IPv6-netwerken. Van het bekijken van gedetailleerde statistieken van specifieke IPv6-adressen tot het configureren van filterregels, deze tool draagt bij aan een geavanceerd en gestroomlijnd netwerkbeheerproces.
\newline

De aanval met ICMPv6 error-pakketten onderstreept opnieuw hoe eenvoudig het is om een dergelijke aanval uit te voeren en de mogelijke gevolgen die dit kan hebben voor een bedrijf. Het verontrustende feit is dat deze aanval kan worden uitgevoerd zonder uitgebreide voorkennis, aangezien online tools direct beschikbaar zijn voor kwaadwillende individuen. Deze aanval heeft het potentieel om aanzienlijke schade toe te brengen aan een organisatie, met name aan slecht geconfigureerde servers die gevoelig zijn voor dergelijke aanvallen.

Het zorgwekkende aspect van deze aanval is dat het mogelijk is om servers direct plat te leggen zonder enig voorafgaand waarschuwingssignaal van een aanval. Dit betekent dat een bedrijf zich mogelijk niet bewust is van de aanval totdat de servers onbereikbaar worden en de bedrijfsactiviteiten ernstig verstoord worden. Dit benadrukt het belang van proactieve maatregelen om de beveiliging van het netwerk te versterken en potentiële aanvallen voor te zijn.
\newline

Gelukkig kan de impact van ICMPv6 error-aanvallen aanzienlijk worden verminderd door het nemen van passende beveiligingsmaatregelen, met name door het configureren van firewalls. Door de kennis en expertise van de netwerkbeheerder te benutten, kunnen firewalls worden geconfigureerd om deze specifieke aanvallen te detecteren en te blokkeren. Hierdoor wordt de kwetsbaarheid van het netwerk verminderd en worden servers beter beschermd tegen potentieel schadelijke verkeersstromen.
\newline

Het is essentieel dat organisaties zich bewust zijn van de risico's van ICMPv6 error-aanvallen en zich inzetten voor een robuuste beveiligingsinfrastructuur. Dit omvat het regelmatig evalueren en verbeteren van de serverconfiguraties, het implementeren van firewalls en andere geavanceerde beveiligingsmaatregelen, en het investeren in de ontwikkeling van het beveiligingsbewustzijn van medewerkers.
Het is duidelijk dat het implementeren van passende beveiligingslagen van cruciaal belang is om de kwetsbaarheid van systemen te verminderen en de integriteit van het netwerk te waarborgen. De resultaten van deze aanval demonstreren het belang van proactieve beveiligingsmaatregelen, zoals SYN-cookies en firewallregels, om potentiële aanvallen te voorkomen of hun impact te minimaliseren. Door te erkennen dat verschillende besturingssystemen verschillende reacties vereisen op TCP-SYN-aanvallen, kunnen beveiligingsteams gerichte maatregelen nemen om de specifieke zwakke punten van elk systeem te versterken. Deze geïntegreerde aanpak draagt bij aan een veerkrachtige beveiligingsinfrastructuur die zich aanpast aan de evoluerende bedreigingslandschappen en zorgt voor een betrouwbaar en veilig netwerk.
\newline

Een belangrijke focus van het onderzoek is het gebruik van opensource tools voor mitigatie van IPv6-aanvallen. Opensource tools bieden vaak flexibiliteit, transparantie en een gemeenschapsgerichte benadering van beveiliging. Door het potentieel van opensource tools te benutten, kunnen beveiligingsexperts oplossingen ontwikkelen die zowel effectief als kostenefficiënt zijn voor kleinere bedrijven.