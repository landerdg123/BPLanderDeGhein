%%=============================================================================
%% Samenvatting
%%=============================================================================

% TODO: De "abstract" of samenvatting is een kernachtige (~ 1 blz. voor een
% thesis) synthese van het document.
%
% Een goede abstract biedt een kernachtig antwoord op volgende vragen:
%
% 1. Waarover gaat de bachelorproef?
% 2. Waarom heb je er over geschreven?
% 3. Hoe heb je het onderzoek uitgevoerd?
% 4. Wat waren de resultaten? Wat blijkt uit je onderzoek?
% 5. Wat betekenen je resultaten? Wat is de relevantie voor het werkveld?
%
% Daarom bestaat een abstract uit volgende componenten:
%
% - inleiding + kaderen thema
% - probleemstelling
% - (centrale) onderzoeksvraag
% - onderzoeksdoelstelling
% - methodologie
% - resultaten (beperk tot de belangrijkste, relevant voor de onderzoeksvraag)
% - conclusies, aanbevelingen, beperkingen
%
% LET OP! Een samenvatting is GEEN voorwoord!

%%---------- Nederlandse samenvatting -----------------------------------------
%
% TODO: Als je je bachelorproef in het Engels schrijft, moet je eerst een
% Nederlandse samenvatting invoegen. Haal daarvoor onderstaande code uit
% commentaar.
% Wie zijn bachelorproef in het Nederlands schrijft, kan dit negeren, de inhoud
% wordt niet in het document ingevoegd.

\IfLanguageName{english}{%
\selectlanguage{dutch}
\chapter*{Samenvatting}

\selectlanguage{english}
}{}

%%---------- Samenvatting -----------------------------------------------------
% De samenvatting in de hoofdtaal van het document

\chapter*{\IfLanguageName{dutch}{Samenvatting}{Abstract}}


Het huidige tijdperk van digitale transformatie heeft geleid tot een steeds groter wordend belang van IPv6, aangezien het internet overgaat naar dit nieuwe protocol. Het biedt een aanzienlijk grotere adresseringsruimte en ondersteunt de groeiende vraag naar IP-adressen. Echter, deze overgang naar IPv6 brengt ook nieuwe uitdagingen met zich mee op het gebied van cybersecurity. Kleinschalige bedrijven, die vaak beperkte middelen en expertise hebben, lopen het risico om kwetsbaar te worden voor potentiële aanvallen. Dit kan ernstige gevolgen hebben, niet alleen voor hun eigen bedrijfsvoering, maar ook voor de veiligheid van hun klanten.
Grondig onderzoek naar IPv6-aanvallen is van primair belang om effectieve veiligheidsmaatregelen te ontwikkelen die deze aanvallen kunnen voorkomen of beperken. In dit streven richt het huidige onderzoek zich specifiek op het begrijpen van de impact en gevolgen van de meest voorkomende IPv6-aanvallen met behulp van een vrij beschikbare opensource tool. Het voornaamste doel van dit onderzoek is om een diepgaand inzicht te verkrijgen in de schadelijke gevolgen van deze aanvallen en de significante invloed ervan op de beveiliging van netwerken en systemen. Daarnaast streeft het onderzoek ernaar om optimale methoden voor mitigatie te identificeren en te bevorderen, met een bijzondere nadruk op het gebruik van opensource tools. Het doel is om de meest effectieve en efficiënte strategieën te ontdekken voor het voorkomen, detecteren en beperken van IPv6-aanvallen.
